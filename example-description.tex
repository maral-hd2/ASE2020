
\subsection{An Example of Test Path and Input Generation}

\begin{figure*}[htb]
%\begin{figure*}[!b]
  \centering
  \begin{minipage}[t]{0.31\textwidth}
    \centering
    \lstset{language=PHP, title=v0.php (original version),
basicstyle=\footnotesize,  tabsize=2}
    \lstinputlisting{figures/v0.php}
  \end{minipage}
  \begin{minipage}[t]{0.31\textwidth}
   \lstset{language=PHP, title=v1.php (modified version),
basicstyle=\footnotesize, tabsize=2}
    \lstinputlisting{figures/v1.php}
  \end{minipage}
    \begin{minipage}[t]{0.35\textwidth}
      \caption*{\\PDG for v0 and v1}
      \vspace*{6pt}
    \includegraphics[width=\textwidth]{figures/flow2.pdf}

  \end{minipage}
  \caption{Path Generation Example Program}
%\vspace*{-5pt}
  \label{fig:example}
\end{figure*}



%We illustrate how we generate test paths using program slices.
%Suppose we have a simple PHP program (v0.php) and its 
%modified program (v1.php) as shown in Figure \ref{fig:example}.
%The rightmost graph shows a PDG for v0.php and v1.php.
%
%In the PDG, the solid lines represent control dependence edges, 
%and the dashed lines represent data dependence edges. 
%As the example shows, statement 5 has changed, so our slicing
%algorithm takes node 5 and variable $a$ ($<$5, $a>$) as a slicing 
%criterion. By performing forward and backward slicing, the slice 
%with respect to $<$5, $a>$ includes nodes 1, 5, 7, and 9 (node 1 
%from backward slicing, and nodes 7 and 9 from forward slicing). 
%
%Having obtained this slice, the path generator creates subpaths 
%starting from the first node in it (in this example, node 1). 
%As the tool walks the path using control flow information, 
%it adds nodes 2 and 3 to a subpath (\{1, 2, 3\}). 
%Once the path generator reaches node 3, it has a choice to 
%explore one of the control successors (4 and 6).
%When the tool analyzes the path containing edge 
%3 $\rightarrow$ 4, it sees that the next 
%control successor is node 5, yielding a subpath of 
%\{1, 2, 3, 4, 5\}. 
%When the tool analyzes the path containing edge 
%3 $\rightarrow$ 6, it discovers that
%there is no path from node 6 to node 5. It then discards
%path \{1, 2, 3, 6\}. 
% 
%Continuing with the remaining edges, the tool produces
%a subpath of \{1, 2, 3, 4, 5, 7, 8\}. Because node 8 
%is another branching node, the tool yields subpaths 
%\{1, 2, 3, 4, 5, 7, 8, 9\} and \{1, 2, 3, 4, 5, 7, 8, 10\}. 
%Edges 8 $\rightarrow$ 9 and 8 $\rightarrow$ 10 are marked as 
%covered by the path generator. The tool then recognizes that 
%subpath \{1, 2, 3, 4, 5, 7, 8, 10\} ends with a node that is 
%outside the impact set. (It is impossible to go from node 10 
%back to nodes \{1, 5, 7, 9\}.) The tool also recognizes that 
%there is no path from node 7 to node 9 that goes through node 10. 
%The tool then discards this path. This process is repeated until 
%all nodes in a slice have been visited.
%
%Having generated subpaths that contain all nodes in a slice,
%the tool walks the program dependence graph from the 
%first node to the beginning of the program (In this case, node 1 
%is at the beginning.) and walks the last node in the path to the 
%end of the program by adding nodes 11 and 12 to it.
%In this example, the tool generates one final path,
%\{1, 2, 3, 4, 5, 7, 8, 9, 11, 12\}.
%Note that, without applying our approach, we need to generate
%four linearly independent paths to test the modified program,
%v1.php. 
%
%The next step is to generate inputs for the created path. 
%First, the constraint collector gathers constraint information
%by analyzing all the path's branching nodes. 
%In our example for path \{1, 2, 3, 4, 5, 7, 8, 9, 11, 12\}, 
%three constraints are collected:  
%\$ POST[`inputA'] $<$ 12 , \$ POST[`inputA']-3 $>$ 7, and 
%\$ POST[`inputB'] $==$ 5. 
%Using a constraint solver, we can choose input values 11 and 
%5 for \$ POST[`inputA'] and \$ POST[`inputB'], respectively.
%Once these values are resolved, the test execution engine
%can simply use them as inputs for the web application to walk 
%the desired path.
%
% 
