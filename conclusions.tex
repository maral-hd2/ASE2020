
\section{Conclusions and Future Work}
\label{sec:conclusions}

To date, researcher and practitioners  
have proposed various techniques 
to reduce the cost of regression testing.
However, existing techniques mainly rely on either dynamic
coverage information or static program analysis, 
which they often have significant
cost of overhead and are imprecise.
In this research, we have introduced a novel 
technique, GTXCrawler, to improve the effectiveness as well 
as reducing the cost of regression testing.
GTXCrawler uses the textual similarity 
of a program source code and returns 
a list of test cases that are
are highly likely covering the modified portion 
of the program.
GTXCrawler does not need any dynamic code coverage 
or static analysis and also it is flexible, scalable, and 
language independent. 
We evaluated our approach using four open-source software projects 
with three other test prioritization techniques.
Our empirical results indicate that the use of the GTXCrawler can reduce 
the the cost of regression testing considerably. 



Because our initial attempt to apply GTXCrawler in the area of 
regression testing has shown promising results, 
we aim to investigate this approach further by 
considering various algorithms (e.g., a hybrid ranking algorithm), 
various characteristics of applications
(e.g., fault history information), 
different application domains (e.g., mobile applications), 
and various testing contexts (e.g., different testing processes or environment). 
For future work, we wish to continue expanding on 
this research by adding a learner algorithm to the framework
that can help the system to be smarter and more efficient by learning 
from it's success and failures.
Moreover, we wish to expand on 
this research as more data becomes available across a 
wider range of applications and different metrics. 
We also want to investigate how this research applies to 
different areas of regression testing such as 
test case selection and reduction.
